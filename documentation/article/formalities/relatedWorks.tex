\section{Related Works}
The literature about our implementation of a ETL testing framework has been limited, as it is still a new area of study. Though different approaches are used, the following papers gives insight to testing ETL processes as a whole.

Mahmoud \cite{dakroryautomated} presents a framework for automatic data centric testing of ETL systems. As opposed to \FW{} it generates its own test cases. Yet testing is not performed upon an implemented ETL as in \FW{}. Instead it uses the ETL design documents in place of the implementation.

ElGamal \cite{elgamal2012towards} analyses current methodologies in the testing of DWs. It is found that no approach provides full coverage of the different components of a DW. The ETL system is one of these components. A methodology framework is presented that emphasizes the use of automation during the testing of components.

In \cite{thomsen2006etldiff} Thomsen describes the ETLDiff framework. Given two ETL systems, the framework is able to compare them and note differences in how they process data from sources to DW. This is used for regression testing to ensure that regression does not occur during further development of an ETL system. Unlike \FW the framework is not able to test for new features.  

Lyet \cite{subuiyer2014} describes a real life example of automating ETL testing in a team of developers. By developing their own test framework, testers can now perform testing earlier in the development cycle and with less effort. This leads to a better quality product and allows for development to be more agile. 

Adding on to this Hegadi \cite{manjunath2012case} empirically evaluates the benefits of automating regression testing using  Informatica. Compared to manual testing. there is a reduction in test cycle time by 50-60\% and a reduction in effort cost by 84\%. 

All these papers mentioned, has provided us with useful hints in setting up our framework, as well as providing us some guidance in determining one what the what the more important forms of testing ETL processes would be. That said, our paper does seem to be among some of the first attempts to build a framework with source to target testing.



