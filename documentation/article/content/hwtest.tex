\section{Overview of \FW{}}
In this section we will give an overview of \FW{} and its major components. The following sections will go into detail about each individual part of the framework.

The purpose of \FW{} is to assist in functional source to target testing of ETL programs developed using pygrametl. As such, testing is performed by asserting about properties of DWs, which have been populated by an ETL program. The focus of testing lies on data loss and business rules. Testers can make assertions about new functionality or perform regression testing using old assertions. Testing is performed at the system level, meaning that the extraction, transformation and loading components have been integrated prior to test. The framework is not designed for exhaustive testing, but rather it is meant to assist in reaching an adequate level of test coverage.

\FW{} contains three major components:
\begin{itemize}
\item\textbf{Reinterpreter}
\item\textbf{DWRepresentation}
\item\textbf{Predicates}
\end{itemize}

The purpose of the reinterpreter is to avoid having users change their pygrametl programs for testing. ETL developers will often hardcode connections to sources and DW's into their ETL program. These contain actual data collected and used by the organization. Because of regulations such data is not available for testers to use. Thus, they have to create their own test sources and DW's. However, they should avoid having to hardcode connections to test data in their programs, as this may be time consuming and error prone. Instead, the reinterpreter allows users to dynamically insert test connections into their ETL program, when performing tests.

The reinterpreter also assists in building a DWRepresentation object from the ETL program under test. DWRepresentation is a class made to contain structural information about the DW being populated. It also allows for easy access of the DW tables and their meta data.

The Predicates are a family of classes used to make assertions about a populated DW. All of them inherit from the Predicate class. Testers make assertions by instantiating a predicate class with the tables of interest along with some other arguments depending on predicate type. Once instantiated, a predicate can be made to check whether the defined assertion hold or not and then report upon it.

In \FW{} a test case is defined as a run of an ETL program on a specific set of test sources and test DW. This is followed by checking and reporting on an assertion through the execution of a predicate. To avoid having to re-run the ETL program often, \FW{} allows for several predicates to be executed after a single run. To create a test case, the tester simply instantiates the Case class as seen below:

\insertcodefile{codeRelated/scripts/Case.py}{Instantiation of a Case object}

The parameters are defined as follows:
\begin{itemize}
\item \textbf{program}: The ETL program under test
\item \textbf{sources}: A list of connection objects to test sources. These are ordered acording to the apperance of the corresponding hard coded connection in the program. 
\item \textbf{predicates}: A list of predicates to execute on the populated DW
\item \textbf{pep249\_module}: The pep249 module used to access the test DW
\item \textbf{program\_is\_path}: Flag indicating whether the program was given as a path or string
\item \textbf{**dw\_conn\_params}: Keyword arguments used to connect to the test DW through the given module. In the case above we use sqlite3 to connect to the DW. The only argument needed to connect is 'database' indicating the location of the DW. In other cases, different arguments may be needed to create a connection.
\end{itemize}

After instantiation of a Case object we call its run() method. This starts off by creating a connection to the test DW. Then the reinterpreter dynamically inserts test sources and DW into the program. The program is then executed. From there we pass over the program again to generate the DWRepresentation. This object is then used to access the DW, when we execute the predicates with it. Once executed we allow the predicates to report to the user.

In the following sections we go into detail about the implementation of the three components discussed in this section.







