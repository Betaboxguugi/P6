\section{pygrametl}\label{sect:pygrametl}
In this section, we give a quick outline of the python package pygrametl, as \FW{} is developed to work in conjunction with this framework. The section is based upon the pygrametl documentation and its source code\cite{pygramSource}.

The package was developed at Aalborg University in Denmark and was released as open source in 2009. It has since then found use in a variety of different systems. It is developed to let users develop ETL programs by coding in python. This stands in contrast to other products on the market, which mainly allow their ETL development through a GUI. By allowing expert developers to use an API instead, it is believed that they will be more productive.

pygrametl provides assistance for the implementation of all three parts of an ETL system. Classes in the datasources.py module allow for the extraction of data from sources. A source can either be a CSV file or an SQL database. Data is always extracted as dictionaries. Each dictionary corresponds to a row from the source, and each key-value pair matches an attribute-value pair of that row. Transformation of the extracted data is supported by the classes found in steps.py. Although users often perform their transformations through normal python. Afterwards, loading is performed using the connectionwrapper class found in init.py and the classes found in tables.py. A connectionwrapper object is used as the default connection to the DW that must be populated. The table.py classes support the use of different types of dimension and fact tables. Once a table class is instantiated it is associated with an actual table in the DW. It can then be used to load data into that table.

Note that pygrametl never sets up sources or DW, it merely provides a way to work with them at an abstract level. They need to be set up separately by developers, and pygrametl must be used in conjunction with a python module that can access them. Such modules are often developed to be used with a specific Database management systems (DBMS). However, pygrametl only works with modules that conform to the PEP 249 standard \cite{pep249}. This standard encourages similarity between modules that access databases. Because of this, pygrametl works with a broad range of DBMS’. To access databases, pygrametl generates SQL code dynamically. The SQL generated is not DBMS specific and uses only standard keywords such as UPDATE, SELECT and JOIN. This is done to ensure compability with the different PEP 249 modules and their related DBMS’. \FW{} aims for the same compatibility and avoids more DBMS specific syntax and functionality when generating SQL code. This also makes accessing metadata on a database or table difficult, as PEP 249 does not enforce a standard way to do this.
