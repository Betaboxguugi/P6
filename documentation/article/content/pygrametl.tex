\section{pygramETL}\label{sect:pygrametl}
In this section we give a quick outline of pygrametl, as \FW{} works in combination with this python package. Knowing about this will help in understanding later parts of this article. 

The package was developed at Aalborg University in Denmark, and it was released as open source in 2009. It has found use in a range of different systems since then. It was developed to let users develop ETLs by coding in python. This stands in contrasts to other products on market, which mainly allow for ETL development through a GUI. By allowing expert developers to use an API instead, they can be more productive.

pygrametl provides an implementation of a set of classes, which can be used to write an ETL system. Source-classes assist in extracting data from different types of sources, such as sql databases and csv files. Data is always extracted as a list of dictionaries. Each dictionary corrresponds to a row, and each key-value pair matches an attribute-value pair for that row. Once such a datastructure has been created, users can perform transformations upon it either using their own functions or the Step-classes provided by pygrametl. Before loading into a DW a ConnectionWrapper object is created. This points to a DW set up by the user. Afterwards data can be inserted into the DW, through instantiations of  Dimension and Fact Table classes provided by pygrametl. These classes support both slowly changing dimensions, as well as snowflaking in the DW.   

Python modules have been developed to communicate with different Database managamenet systems (DBMS). Such modules must be used in conjunction with pygrametl to set up and communicate with databases. However, pygrametl only works with modules that conform to the PEP 249 standard \cite{pep249}. This standard encourages similarity between modules that access databases. Because of this, pygrametl works with a broad range of DBMSs. This forces the package to access databases in a very generic manner. Often this means that the SQL code dynamicly generated by pygramet is very limited. This is necessary as different DBMSs use different implementations of SQL. 
    
