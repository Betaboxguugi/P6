\section{pygramETL}\label{sect:pygrametl}
In this section we give a quick outline of pygrametl, as \FW{} works in combination with this python package. The section is based upon the pygramet documentationl and its source code\cite{pygramSource}.   

The package was developed at Aalborg University in Denmark and was released as open source in 2009. It has found use in a range of different systems since then. It was developed to let users develop ETLs by coding in python. This stands in contrast to other products on market, which mainly allow for ETL development through a GUI. By allowing expert developers to use an API instead, they can be more productive.

pygrametl provides assitance for the implementation of all three parts of an ETL system. Classes in the datasources.py module allow for the extraction of data from sources. A source may either be a CSV file or an SQL database. Data is always extracted as a list of dictionaries. Each dictionary corrresponds to a row from the source, and each key-value pair matches an attribute-value pair of that row. Transformation of the extracted data is supported by the classes found in steps.py. Although users often perform their transformations in other ways. Afterwards, loading is performed using the connectionwrapper class found in init.py and the classes found in tables.py. A connectionwrapper object is used as the default connection to the DW that must be populated.  The table.py classes support the use of different types of dimension and fact tables. Once a table class is instantiated it is associated with an actual table in the DW. It can then be used to load data into that table. 

Note that pygrametl never sets up sources or DW, it merely provides a way to work with them at an abstract level. These need to be set up seperatly by developers, and pygrametl must be used in conjunction with a python module that can access them. Such modules are often developed to be used with a specefic  Database managamenet systems (DBMS) . However, pygrametl only works with modules that conform to the PEP249 standard \cite{pep249}. This standard encourages similarity between modules that access databases. Because of this, pygrametl works with a broad range of DBMSs. This forces the package to access databases in a very generic manner. As such, the SQL code dynamicly generated by pygramet is very limited, as different DBMSs use varying implementions of SQL. \FW{} has the same restrictions placed upon it, and thus has its limitations when communicating with a database through a PEP249 connection. Accessing meta data on a database or table is especially difficult, as PEP249 does not enforce a standard way to do this. 
    
