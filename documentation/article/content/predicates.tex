\section{Predicates}
\todo[inline]{Skriv introduktion.}
In the following section...

\subsection{ColumnNotNullPredicate}
ColumnNotNullPredicate asserts that all entries in a specified list of columns, within a table, does not contain null. Allowing a user to ensure that the data within a table has all entries filled after a use of a function or insertion of data.

The ColumnNotNullPredicate is instantiated as follows, with description of the parameters below.

\insertcodefile{ColumnNotNullPredicate.py}{Instantiation of ColumnNotNullPredicate}

\begin{description}
\item [table\_name] a string containing the name of the specific table from the DW, that the test must run on. 
\item [column\_names] a list of strings or a string containing the name(s) of the column(s), which specifies which column(s) the predicate should test within the table
\end{description}

ColumnNotNullPredicate iterates over every row in the specified table. In each row it then iterates over each element in the column(s) specified. Through this later iteration it insures that each element is not null. If a null is found, its row is appended to a list, which is displayed as part of the report object, which also will report the result of the test, as false.


\subsection{CompareTablePredicate}
CompareTablePredicate asserts that two tables are either equal or that one is a subset of the other. It is made to allow users to give an expected table to compare with the actual one populated by their pygrametl program. It checks if an identical row exists in the actual table, for each row in the expected table. The CompareTablePredicate is instantiated as follows, with description of the parameters below.

\insertcodefile{CompareTablePredicate.py}{Instantiation of CompareTablePredicate}

\begin{description}
\item [actual\_name] is a string containing the name of the table from the DW, to be compared with the given \textit{expected\_table}. It is used to index the DWRepresentation and then retrieve the actual table.
\item [expected\_table] is a list of dictionaries, representing a database table, and is used to check whether it is equal to, or a subset of the actual table.
\item [ignore\_atts] is a list of string containing the names of the attributes you wish to ignore when comparing the two tables. E.g. one could imagine that it would be nice to not compare on surrogate keys, as they might change from run to run of the pygrametl program.
  \item [subset] is a boolean, if \textit{true} the expected table only has to be a subset of the actual table, else they have to be equal.
\end{description}

For the instantion above, we assert that BookDim contains the 2nd version of the Hobbit. We ignore the bid attribute for the compare, as it is a surrogate key. Surrogate keys are generated by software, and the user need not concern its contents. 

CompareTablePredicate works by using the filterfalse function from the standard python package \textit{itertools}. It uses this to take the relative complement of the two tables, and we assert that the predicate holds true for the case of \textit{subset} equal \textit{true} if,

\[ A \backslash E \equiv \emptyset \]

\noindent which implies that $A \subseteq E$, where A is the actual table and E is the expected table.

And for \textit{subset} equal \textit{false} if,

\[ A \backslash E \cup E \backslash A \equiv \emptyset \]

\noindent which implies that $A \equiv E$, where A is the actual table and E is the expected table.


\subsection{NoDuplicateRowPredicate}
NoDuplicateRowPredicate asserts that a given table contains only unique entries. This is often important, when we want to ensure no duplicates of primary keys. Using the predicate, the tester may also indicate, what columns should be considered as part of the assertion. As such, we may specify that a whole row does not have to be unique, only a subset of its columns. This is useful when asserting unique keys or lookup attributes. In \Cref{NoDuplicateRowPredicate.py} is an example of how the predicate can be instantiated.

\insertcodefile{NoDuplicateRowPredicate.py}{Instantiation of NoDuplicateRowPredicate}

In the instantiation displayed, the predicate asserts that each entry in the aid column of AuthorDim is unique. This is important as aid is the primary key to the table.
NoDuplicateRowPredicate will in this case generate the SQL query shown in \Cref{SQLNoDuplicateRowPredicate.sql}, to satisfy the previous assertion:

\insertcodefileSQL{SQLNoDuplicateRowPredicate.sql}{SQL query generated from  \Cref{NoDuplicateRowPredicate.py}}

This query groups together all rows on aid, and fetches groups with more than one member in AuthorDim. These groups are duplicates. If any are found the assertion fails, and the duplicates are reported to the tester.



\subsection{ReferentialIntegrityPredicate}
ReferentialIntegrityPredicate asserts that there is referential integrity between all tables of  a DW. This means that each row of a table with a foreign key, has a corresponding row in the table to which it is referring. Referential integirty is not always upheld during load, so a tester may want to assert this property. This predicate takes no parameters for its instatiation, and it is simply run on the DWRepresentation corresponding to the test DW .

ReferentialIntegrityPredicate makes use of the DWRepresentation object's \emph{refs} field, which is similar to \emph{refs} used in pygramETL's snowflaked dimensions. \emph{refs} is a dictionary with referring tables as keys and lists of tables as values as described in \cref{sect:interdatarep}. ReferentialIntegrityPredicate uses this to create a dictionary of dictionaries that holds more information on the references. These inner dictionaries will have foreign keys as their key with the referenced table as the value. In this way it is easy to look up what the foreign keys are called, and what they refer to. We can theno iterate over the tables, their rows, and comparing their keys that they have in common. After iterating between a referring table and a referred table, the checks are made again by iterating over the referred table's rows first. This is done because the new \emph{refs} have no information on the tables a table is referred by, only what it refers to. If the predicate fails to find a corresponding id for a row in another table, it takes note of the two tables and the key for reporting. Currently works with facttables and snowflaked dimensions.
\subsection{RowCountPredicate}
RowCountPredicate asserts if a table has the specified number of rows. This can be useful in situations where the user knows how many rows should be in a given table, after a function or transformation has been done upon it. RowCountPredicate is instantiated as follows, with description of the parameters below.

\insertcodefile{RowCountPredicate.py}{Instantiation of RowCountPredicate}

\begin{description}
\item [number\_of\_rows] a integer representing the specified number of rows we are looking for
\end{description}

RowCountPredicate iterates over each row in a table, counting them and adding them to a list. The length of the list is then compared to number provided by number\_of\_rows. If equal, the test is successful, if not equal, the test return false and reports how many rows are actually in the table.


\subsection{RuleColumnPredicate}
RuleColumnPredicate is similar to RuleRowPredicate in \cref{RCP}, even taking the same parameters as input and being performed mainly through python code. The difference is that RuleColumnPredicate allows for the assertion of business rules that apply to sets of columns. It is instantiated as shown in \Cref{RuleColumnPredicate.py}

\insertcodefile{RuleColumnPredicate.py}{Instantiation of RuleColumnPredicate}

In this case, we give no\_forgotten\_cities as the constraint function. The predicate asserts that no city in the AuthorDim table is named Constantinople.

During execution, each column specified by column\_name is fetched from the database in its entirety. The columns are then supplied to the constraint\_function along with the additional parameters in constraint\_args. Note that in contrast to RuleRowPredicate, we only call the function a single time, since it takes entire columns as input. The assertion made by the predicate holds, based on whether constraint\_function returns True or False.
\subsection{TabelPredicate}
TabelPredicate 
It shares its usefulness with RuleColumnPredicate in \cref{RCP}

\insertcodefile{codeRelated/scripts/TabelPredicate.py}{Instantiation of TabelPredicate}

\begin{description}
\item [table\_name] a string containing the name of specific table from as data warehouse. 
\item [constraint\_function] a predicate that represent the constraint which need to be tested. When called it must return either true, if the constraint is fulfilled, or false if it's not.
\item [column\_names] a list of strings or a string containing the name(s) of the column(s), which specifies which column(s) the predicate should test within in the table
\item [return\_list] a bool, if false then all elements from the column(s) is send as arguments to the constraint function, if true they are send as a list instead.
\end{description}

TabelPredicate iterates...

\subsection{FunctionalDependencyPredicate}
FunctionalDependencyPredicate asserts that a table holds a certain functional dependency. This can prove useful to check if a DW holds certain hierarchical properties, i.a.. FunctionalDependencyPredicate is instantiated as follows, with a description of its parameters shown in \Cref{FunctionalDependencyPredicate.py}.

\insertcodefile{FunctionalDependencyPredicate.py}{Instantiation of FunctionalDependencyPredicate}

\begin{description}
\item [alpha] attributes being the alpha of the given functional dependency. Given as either a single attribute name, or a tuple of attribute names.
\item [beta] attributes which are functionally dependent on alpha. Given as either a single attribute name, or a tuple of attribute names.
For example with alpha as \texttt{('a', 'b')} and beta as  \texttt{'c'} we get the functional dependency: $a,\ b \rightarrow c$.
\end{description}

In our instantiation above, we use the predicate to assert that there is a functional dependency between a book's title and its author. If this assertion holds, it means that no book is written by more than one author.

In \Cref{SQLFunctionalDependency.sql} we see the query generated in this case. Here we join the table with itself on alpha, then return all distinct rows in which alpha can not be used to determine beta. If the query returns any rows, the assertion fails and the rows are reported to the tester.    

\insertcodefileSQL{SQLFunctionalDependency.sql}{SQL query generated from
\Cref{FunctionalDependencyPredicate.py}}


\insertcodefile{codeRelated/scripts/ColumnNotNullPredicate.py}{Instantiation of ColumnNotNullPredicate}

\begin{description}
\item [table\_name] a string containing the name of the specific table from a data warehouse, that the test must run on. 
\item [column\_names] a list of strings or a string containing the name(s) of the column(s), which specifies which column(s) the predicate should test within the table
\end{description}

ColumnNotNullPredicate iterates over every row in the specified table. In each row it then iterates over each element in the column(s) specified. Through this later iteration it ensures that each element is not null. If null is found, the row it was found in is appended to a list, which is displayed as part of the report object, which also will report the result of the test, as false.

\subsection{CompareTablePredicate}
CompareTablePredicate asserts that two tables are either equal or that one is a subset of the other. It is made to allow users to give an expected table to compare with the actual one populated by their pygrametl program. It checks if an identical row exists in the actual table, for each row in the expected table. The CompareTablePredicate is instantiated as follows, with description of the parameters below.

\insertcodefile{codeRelated/scripts/CompareTablePredicate.py}{Instantiation of CompareTablePredicate}

\begin{description}
\item [actual\_name] is a string containing the name of the table from your datawarehouse, you would like to compare with the given \textit{expected\_table}. It is used to index our datawarehouse representation and then retrieve the actual table.
\item [expected\_table] is a list of dictionaries, representing a database table, and is used to check whether it is equal to, or a subset of the actual table.
\item [ignore\_atts] is a list of string containing the names of the attributes you wish to ignore when comparing the two tables. E.g. one could imagine that it would be nice to not compare on surrogate keys, as they might change from run to run of the pygrametl program.
  \item [subset] is a boolean, if \textit{true} the expected table only has to be a subset of the actual table, else they have to be equal.
\end{description}

CompareTablePredicate works by using the filterfalse function from the standard python package \textit{itertools}. It uses this to take the relative complement of the two tables and we assert that the predicate holds true for the case of \textit{subset} equal \textit{true} if,

\[ A \backslash E \equiv \emptyset \]

\noindent which implies that $E \subseteq A$, where A is the actual table and E is the expected table.

And for \textit{subset} equal \textit{false} if,

\[ A \backslash E \cup E \backslash A \equiv \emptyset \]

\noindent which implies that $A \equiv E$, where A is the actual table and E is the expected table.

\subsection{RuleColumnPredicate}

RuleColumnPredicate asserts that all entries of each row in a specified list of columns within a table, complies with a constraint, given trough a predicate made by the user. This allows the user a lot of freedom in their tests, as they can test each row of elements in a given table, in any way they wish. Ensuring the data complies with the requirements they have set for it. The RuleColumnPredicate is instantiated as follows, with description of the parameters below.

\insertcodefile{codeRelated/scripts/RuleColumnPredicate.py}{Instantiation of RuleColumnPredicate}

\begin{description}
\item [table\_name] a string containing the name of specific table from as data warehouse. 
\item [constraint\_function] a predicate that represent the constraint which need to be tested. When called it must return either true, if the constraint is fulfilled, or false if it's not.
\item [column\_names] a list of strings or a string containing the name(s) of the column(s), which specifies which column(s) the predicate should test within in the table
\item [return\_list] a bool, if false then all elements from the column(s) is send as arguments to the constraint function, if true they are send as a list instead.
\end{description}

RuleColumnPredicate iterates over every row in the specified table. For each row, it iterates over the column(s) specified, where it takes each element from each column(s) and calls the constraint function on those elements as either arguments or a list, depending on what the user has specified. If the constraint function returns false on a row, then the row is appended to a list, which is displayed as part of the report object, which also will report the result of the test, as false. RuleColumnPredicate does not support constraints with a variable number of arguments.

\subsection{NoDuplicateRowPredicate}

NoDuplicateRowPredicate a table does not have 2 or more rows entries which are a duplicate of each other. \todo[inline]{this is helpful because...(include something about unique keys)}

\insertcodefile{codeRelated/scripts/NoDuplicateRowPredicate.py}{Instantiation of NoDuplicateRowPredicate}

\begin{description}
\item [table\_name] a string containing the name of specific table from as data warehouse. 
\item [column\_names] a list of strings or a string containing the name(s) of the column(s), which specifies which column(s) the predicate should test within in the table. Default for column\_names is null, which has some interaction with column\_names\_exclude.
\item [column\_names\_exclude] a bool, if true, then columns\_names specifies which columns the predicate will exclude to look at when comparing. Default is None. Note if columns\_names is null then column\_names\_exclude will be set to true if false too.
\item [verbose] a bool, if this is set to true information from each step in NoDuplicateRowPredicate is printed. Used for debugging.
\end{description}

NoDuplicateRowPredicate chooses a row from the specified table and checks whether it's unique from each of the remaining rows in the table, and does so for every row except the last. Through each iteration the chosen row is popped from the table first, so that each iteration will be smaller from the last. When the rows are checked against each other, each specified column (or if columns were excluded, each non-excluded columns) have their values compared between the rows. If at least one of these values are unique between the rows, the two rows are not duplicates. It then stops checking values for the two rows and moves to check the next row. If the predicate fails to find unique values between the rows, it attempts to add the duplicate row to a set of duplicate rows, which is then shown in the report. Potential problems arise when columns are somehow specified wrongly, as the predicate will collect None type values from non-existent columns and check them against each other, which are then seen as duplicates. Another problem is when primary key values are checked as well, as these should never be duplicates, but the rest of the attributes may still be duplicates between two rows, and thus that information is missed, the programmer needs to be aware of this.

\subsection{ReferentialIntegrityPredicate}
ReferentialIntegrityPredicate asserts that all table entries in all tables in the given datawarehouse connection, have referential integrity. This means that each row of a table with a foreign key, has a corresponding row in the table to which it is referring and vice versa. A user may always want to have full integrity, and this is a good way to check that integrity has not been violated. This predicate takes no parameters, it is enough to simply create one instance and use it for any number of cases and with any number number of datawarehouse connections.

ReferentialIntegrityPredicate makes use of the DWRepresentation object's \emph{refs} field, which is similar to \emph{refs} used in pygramETL's snowflaked dimensions. \emph{refs} is a dictionary with referring tables as keys and lists of tables as values. ReferentialIntegrityPredicate uses this to make a dictionary of dictionaries that holds more information on the references. Now these inner dictionaries will have foreign keys as their key with the referenced table as the value. In this way it is easy to look up what the foreign keys are called and what they refer to. It is then quick to iterate over the tables, their rows, and comparing their keys that they have in common. After iterating between a referring table and a referred table, the checks are made again by iterating over the referred table's rows first. This is done because the new \emph{refs} have no information on the tables a table is referred by, only what it refers to. If the predicate fails to find a corresponding id for a row in another table, it takes note of the two tables and the key for reporting. Currently works with facttables and snowflaked dimensions.
