\section{Predicates}\label{sect:pred}
This section will describe each of the predicate classes found in \FW. These are used to assert specific properties of a DW during test. Each subsection will describe the purpose of a predicate, how it is instantiated and how it functions. Each predicate is related to \Cref{fig:exdw} as to show, how it may be used in practice. In the implementations we have tried to use as much SQL as possible, where it made sense to do so. Working with data close to the database is much faster than treating representation objects within Python. One complication with this approach is that different DBMS' usable
with \FW{} use different implementations of SQL. Thus we risk damaging the compatibility of \FW{} by using SQL. With this in mind, we restrict the predicates to use only the most standard SQL functionality. The SQL queries in question are dynamically generated by the predicates, depending on how they are instantiated. Across the predicates these queries often return all rows found to be faulty according to the assertion. If no rows are returned, a successful assertion is reported. If rows are returned, a failed assertion is reported along with the faulty rows. In general, more in-depth reporting upon predicates only occurs, when they fail.   

For instantiation of different predicate types some common parameters are used. These are presented here, as not to repeat the information later.
\begin{description}
\item [table\_name] a string with the name of a table from the DW. Using the get\_data\_representation method from DWRepresentation, this is used to fetch the corresponding table.Can also be given as a list of strings, if a natural join of tables is wanted insead.
\item [column\_names] a string or list indicating one or more attributes from the table. If the attribute is None, we fetch all attributes for each row. If not, the data fetched depends on column\_names\_exclude.
\item [column\_names\_exclude] a boolean that indicates, which attributes to work on. If false, we use the attributes as described in column\_names. If true we use all attributes not in column\_names. Default is false. 
\end{description}

\subsection{RowCountPredicate}
RowCountPredicate asserts if a table has the specified number of rows. This can be useful in situations where the user knows how many rows should be in a given table, after a function or transformation has been done upon it. RowCountPredicate is instantiated as follows, with description of the parameters below.

\insertcodefile{RowCountPredicate.py}{Instantiation of RowCountPredicate}

\begin{description}
\item [number\_of\_rows] a integer representing the specified number of rows we are looking for
\end{description}

In our instantiation above, we assert that CountryDim has 200 entries.  

RowCountPredicate iterates over each row in a table, counting them and adding them to a list. The length of the list is then compared to number provided by number\_of\_rows. If equal, the test is successful, if not equal, the test return false and reports, how many rows are actually in the table.


\subsection{ColumnNotNullPredicate}
For a set of specified columns within a table, ColumnNotNullPredicate lets testers assert that no member of the given columns are null. While nulls may be allowed to occur within a table, they could indicate data loss. It may also be the case that not null constraints are placed upon certain columns. This is the case for primary keys. The ColumnNotNullPredicate is instantiated as exemplified below:

\insertcodefile{ColumnNotNullPredicate.py}{Instantiation of ColumnNotNullPredicate}

For instantiation, we assert that no entry to the column bid in the BookDim is null. This is important to check, as bid is a primary key to the dimension. 
ColumnNotNullPredicate functions in this case by issuing the following SQL query:

\insertcodefileSQL{SQLColumnNotNullPredicate.sql}{}

This query will show all rows which contain null in the column bid. If any rows are found, the assertion fails and all found rows are reported to the user. Otherwise it succeeds.



\subsection{NoDuplicateRowPredicate}
NoDuplicateRowPredicate asserts that a table contains only unique entries. This is often important, when we want to ensure no duplicates of primary keys. Using the predicate, the tester may also indicate, what columns should be considered as part of the assertion. As such, we may specify that a whole row does not have to be unique, only a subset of its columns. This is useful when asserting unique keys or lookup attributes. Below is an example of how the predicate is instantiated:
\insertcodefile{NoDuplicateRowPredicate.py}{Instantiation of NoDuplicateRowPredicate}

In the instantiation displayed, the predicate asserts that  each entry in the aid column of AuthorDim is unique. This is important as aid is the primary key to the table. 
NoDuplicateRowPredicate will in this case issue the following SQL query, to satisfy the previous assertion:

\insertcodefileSQL{SQLNoDuplicateRowPredicate.sql}{}

The select portion of the predicate counts how many of each row there are. Then it group those rows sowe are only shown rows with are duplicates, 2 or more, within the specified columns.


\subsection{ReferentialIntegrityPredicate}
\todo[inline]{Yet to be written...}
\subsection{FunctionalDependencyPredicate}
\todo[inline]{Yet to be written...}
\subsection{SCDVersionPredicate}\label{SCD}

This predicate allows testers to check if a given entry in a table has an asserted largest version. This predicate only works on SCDType2Representation objects, which represent type 2 slowly changing dimensions. SCDVersionPredicate is instantiated with parameters as depicted below.

\insertcodefile{SCDVersionPredicate.py}{Instantiation of RuleColumnPredicate}

\begin{description}
\item [entry] A dictionary pairing lookup attributes from the table with values. This indicates a unique entry in the table, which may have several different versions. 
\item [version] A number indicating the largest version that the entry is asserted to have.
\end{description}

In the instantiation, we use the predicate to assert that The Hobbit's highest version number in BookDim is 10.

The predicate works by iterating over each row of the table. Once we find a row that matches the test entry, we check whether its version is bigger than any found before. If yes, we update the value of our largest version. After processing the table, we try to equate the largest version with the one asserted by the tester. If they are equal, the predicate succeeds, and it fails otherwise.   

\subsection{CompareTablePredicate}
CompareTablePredicate asserts that two tables are either equal or that one is a subset of the other. It is made to allow users to give an expected table to compare with the actual one populated by their pygrametl program. It checks if an identical row exists in the actual table, for each row in the expected table. The CompareTablePredicate is instantiated as follows, with description of the parameters below.

\insertcodefile{codeRelated/scripts/CompareTablePredicate.py}{Instantiation of CompareTablePredicate}

\begin{description}
\item [actual\_name] is a string containing the name of the table from a datawarehouse, to be compared with the given \textit{expected\_table}. It is used to index the datawarehouse representation and then retrieve the actual table.
\item [expected\_table] is a list of dictionaries, representing a database table, and is used to check whether it is equal to, or a subset of the actual table.
\item [ignore\_atts] is a list of string containing the names of the attributes you wish to ignore when comparing the two tables. E.g. one could imagine that it would be nice to not compare on surrogate keys, as they might change from run to run of the pygrametl program.
  \item [subset] is a boolean, if \textit{true} the expected table only has to be a subset of the actual table, else they have to be equal.
\end{description}

CompareTablePredicate works by using the filterfalse function from the standard python package \textit{itertools}. It uses this to take the relative complement of the two tables and we assert that the predicate holds true for the case of \textit{subset} equal \textit{true} if,

\[ A \backslash E \equiv \emptyset \]

\noindent which implies that $A \subseteq E$, where A is the actual table and E is the expected table.

And for \textit{subset} equal \textit{false} if,

\[ A \backslash E \cup E \backslash A \equiv \emptyset \]

\noindent which implies that $A \equiv E$, where A is the actual table and E is the expected table.

\subsection{RuleRowPredicate}\label{RCP}
RuleRowPredicate is used to assert that every row of a table complies with some user-defined constraint. Thus, this predicate allows users a lot of flexibility in how they test business rules for individual rows. The RuleRowPredicate is instantiated as shown below:

\insertcodefile{RuleRowPredicate.py}{Instantiation of RuleRowPredicate}

The parameters comprise of:
\begin{description}
\item [constraint\_function] a function that represent the user constraint. It must return a boolean, indicating whether a given row conforms to the constraint.
\item [constraint\_args] a list of additional arguments given to the constraint\_function.
\end{description}

With column\_names we define, which row attributes that constraint\_function should get as input. As we iterate over each row, we call constraint\_function. The function receives the defined row attributes as parameters along with those from constraint\_args. If the function returns false for a row, the assertion did not hold.

In the instantiation above, we use the initials\_id as our constraint function. Using the predicate as a whole, we assert that every author's primary key corresponds to their initials. 

\subsection{RuleColumnPredicate}
RuleColumnPredicate is a variation of the RuleRowPredicate in \cref{RCP}. The difference being that RuleColumnPredicate allows the user to check their constraint against an the entire column at once, instead of going through each row as RuleRowPredicate does. The input to arguments of the constraint is by default a list, each representing the column(s) in the order they are specified. However if return\_list is set to false, the input for the constraint is either the sum if the column contains integers, or the join of all the strings if the column contains strings. Which can be useful in some edge cases, such as testing the total sum of 2 columns, is equal to sum of a 3'rd column. TablePredicate is instantiated as follows, with description of the parameters below

\insertcodefile{RuleColumnPredicate.py}{Instantiation of RuleColumnPredicate}

\begin{description}
\item [constraint\_function] a predicate function that represent the constraint which need to be tested. When called it must return either true, if the constraint is fulfilled, or false if it's not.
\item [constraint\_input\_list] a bool, if true a list for each column specified, will be given to the constraint function as input. If false, the predicate will sum up all numerics, or join all strings depending on what the column. Once done for all the columns, it will be given to the constraint function as input. Default is true.
\end{description}

RuleColumnPredicate iterates over every row in the specified table. For each row, it iterates over the column(s) specified, where it takes each element from each column(s) and calls the constraint function on those elements as either arguments or a list, depending on what the user has specified. If the constraint function returns false on a row, then the row is appended to a list, which is displayed as part of the report object, which also will report the result of the test, as false. RuleColumnPredicate does not support constraints with a variable number of arguments.





