\section{Related Work}\label{sect:RelatedWork}
In this section, we give a brief overview of the literature related to the work discussed in this paper. A lot of the literature discusses what methodology to use when testing ETL systems. While other articles focus on the development of frameworks to assist in testing. Both types of work are presented below.

ElGamal, et al., \cite{elgamal2012towards} analyses current methodologies in the testing of DWs. It is found that no approach provides full coverage of the different components of a DW. The ETL system is considered one of these components. A methodology is presented that emphasizes the use of automation during the testing of components.

Iyer \cite{subuiyer2014} describes a real life example of automating ETL testing in a team of developers. By developing their own test framework, testers are able to perform testing earlier in the development cycle and with less effort. This leads to a better quality product and allows for development to be more agile. However, they focused more on methodology, and they did not discuss their framework in detail. Adding on to this Manjunath, et al., \cite{manjunath2012case} empirically evaluates the benefits of automating regression testing using Informatica. Compared to manual testing, they find that there is a reduction in test cycle time by 50-60\% and a reduction in effort cost by 84\%.

Dakrory, et al., \cite{dakroryautomated} presents a framework for automatic data centric testing of ETL systems. As opposed to \FW{}, it generates its own test cases. Yet testing is not performed upon an implemented ETL as in \FW{}. Instead it uses the ETL design documents in place of the implementation.

In \cite{thomsen2006etldiff} Thomsen and Pedersen describes the ETLDiff framework. Given two ETL systems, the framework is able to compare them, and note differences in how they process data from sources to DW. This is used for regression testing to ensure that regression does not occur during further development of an ETL system. Unlike \FW{} the framework is not able to test for new features only regression.

The papers mentioned above indicate that automated tests have a central place in the field of ETL testing. There is a consensus that its usage can be very beneficial to the development of ETL systems. The literature does not cover any framework, which can be used to test new features of an implemented ETL system. As this is one of the features of \FW{}, this work seems to hit a niché in the research. As discovered in \cref{intro}, we do however see systems on market sharing this feature with \FW{} but none that are specialized for use in conjunction with pygrametl.