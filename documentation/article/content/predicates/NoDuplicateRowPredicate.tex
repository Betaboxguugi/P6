\subsection{NoDuplicateRowPredicate}
NoDuplicateRowPredicate asserts that a table contains only unique entries. This is often important, when we want to ensure no duplicates of primary keys. Using the predicate, the tester may also indicate, what columns should be considered as part of the assertion. As such, we may specify that a whole row does not have to be unique, only a subset of its columns. This is useful when asserting unique keys or lookup attributes. Below is an example of how the predicate is instantiated:
\insertcodefile{NoDuplicateRowPredicate.py}{Instantiation of NoDuplicateRowPredicate}

In the instantiation displayed, the predicate asserts that  each entry in the aid column of AuthorDim is unique. This is important as aid is the primary key to the table. 
NoDuplicateRowPredicate will in this case issue the following SQL query, to satisfy the previous assertion:

\insertcodefileSQL{SQLNoDuplicateRowPredicate.sql}{}

The select portion of the predicate counts how many of each row there are. Then it group those rows sowe are only shown rows with are duplicates, 2 or more, within the specified columns.

