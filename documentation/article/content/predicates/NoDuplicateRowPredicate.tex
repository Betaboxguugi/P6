\subsection{NoDuplicateRowPredicate}

NoDuplicateRowPredicate asserts that a table does not have two or more rows which are a duplicate of each other. It can be specified to only look at certain columns, where it will consider the columns specified as the rows it looks upon. This can be helpful in routing out problems when applying joins, as duplicates can occur in one or more columns when doing so. Otherwise it can also be used after creation of a table or insertion of data into one, as this is also a common area where duplicates may be found if something were done improperly.
 
\insertcodefile{NoDuplicateRowPredicate.py}{Instantiation of NoDuplicateRowPredicate}

\begin{description}
\item [table\_name] a string containing the name of specific table from the DW. 
\item [column\_names] a list of strings or a string containing the name(s) of the column(s), which specifies which column(s) the predicate should test within in the table. Default for column\_names is null, which has some interaction with column\_names\_exclude.
\item [column\_names\_exclude] a bool, if true, then columns\_names specifies which columns the predicate will exclude to look at when comparing. Default is None. Note if columns\_names is null then column\_names\_exclude will be set to true if false too.
\end{description}

NoDuplicateRowPredicate iterates through each row as tuples and inserts them into a hashtable(dictionary). If the tuple already exists in the hashtable, it is a duplicate and is saved to a list for reporting.
Potential problems arise when primary key values are checked as well, as these should never be duplicates, but the rest of the attributes may still be duplicates between two rows, and thus that information is missed, the programmer needs to be aware of this.
