\subsection{FunctionalDependencyPredicate}
FunctionalDependencyPredicate asserts that a table, or natural join of tables, holds a certain functional dependency. This can prove useful to check if a DW holds certain hierarchical properties i.a. FunctionalDependencyPredicate is instantiated as follows, with a description of its parameters below.

\insertcodefile{FunctionalDependencyPredicate.py}{Instantiation of FunctionalDependencyPredicate}

\begin{description}
\item [tables] a list of table names, that will get a natural join between them. Of which the result is what the predicate runs on. 
\item [attributes] attributes which are depended upon by other attributes. Given as either a single attribute name, or a tuple of attribute names.
\item [dependent\_attributes] attributes which are functionally dependent on the former attributes. Given as either a single attribute name, or a tuple of attribute names.

For example with the attributes \texttt{('a', 'b')} and the dependent attribute \texttt{'c'} we get the functional dependency: $a,\ b \rightarrow c$.
\end{description}

In our instantiation above, we use the predicate to assert that there is a functional dependency between a book's title and its author. If this assertion holds, it means that no book is written by more than one author. 

Below in \cref{SQLExample.sql} is an example of the query used in the predicate to find rows that are not in compliance with the functional dependency. Given the instantiation above, the FunctionalDependencyPredicate will create this query. 
\insertcodefileSQL{SQLFunctionalDependency.sql}{SQL example for FunctionalDependencyPredicate}

%\insertcodefile{FunctionalDependencyPredicate_code.py}{Main functionality of FunctionalDependencyPredicate} 
