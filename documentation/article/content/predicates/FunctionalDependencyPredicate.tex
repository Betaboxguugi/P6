\subsection{FunctionalDependencyPredicate}
FunctionalDependencyPredicate asserts that a table, or natural join of tables, holds certain functional dependencies. This can prove useful to check if your DW holds certain hierachical properties i.a.. FunctionalDependencyPredicate is instantiate as follows, with a description of its parameters below.

\insertcodefile{FunctionalDependencyPredicate.py}{Instantiation of FunctionalDependencyPredicate}

\begin{description}
\item [tables] a list of table names, that will be natural joined together. Of which the result is what the predicate runs on. 
\item [fds] a list of tuples, where each tuple descripes a functional dependency. E.g. \texttt{('a', 'b')} means that \texttt{b} is functional dependent on \texttt{a}, i.e. $a \rightarrow b$.
\item [ignore\_None] a boolean, if true then we ignore $None \rightarrow value$ dependencies. This could be useful e.g. the case where we have a functional dependency between cities and countries, i.e. $city \rightarrow country$. Here it would be possible to imagine a scenario where we have two people in different countries, that both have the city None. And as such we can't have a functional dependency for None values here.
  
\end{description}

<<<<<<< HEAD
Below in \cref{codeRelated/scripts/FunctionalDependencyPredicate_code.py} is the main functionality of FunctionalDependencyPredicate written in Python, along with a description. 
=======
Below in \cref{FunctionalDependencyPredicate_code.py} is the main functionality of FunctionalDependencyPredicate written in Python.
>>>>>>> e872ef460683212f5015dfb71724f93619cf7ff8

\insertcodefile{FunctionalDependencyPredicate_code.py}{Main functionality of FunctionalDependencyPredicate} 

\begin{description}
\item [line 1] We create a hash table for each functional dependency given
\item [line 4] We iterate the joined table
\item [line 5] For each row we iterate over each functional dependency
\item [line 6-7] We get the value of the attributes in the functional dependency for this row.
\item [line 9-10] If we should ignore None and if \texttt{x} is None we do not do anything.
\item [line 11-12] If \texttt{x} is a key of the functional dependencies hash table, and if \texttt{y} is not the same value as the hash of \texttt{x}.  Then we treat this row as errorneous and append \texttt{elements} with it.
\item [line 13-14] If \texttt{x} is a key of the hash table and if the hash of \texttt{x} has the same value as \texttt{y}. This means that the functional dependency holds and that we do nothing.
\item [line 15-16] If \texttt{x} is not a key of the hash table, then we add it to the table and we let it hash to the value of \texttt{y}.  
\end{description}
