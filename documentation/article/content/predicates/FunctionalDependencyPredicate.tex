\subsection{FunctionalDependencyPredicate}
FunctionalDependencyPredicate asserts that a table holds a certain functional dependency. This can prove useful to check if a DW holds certain hierarchical properties, i.a.. FunctionalDependencyPredicate is instantiated as follows, with a description of its parameters shown in \Cref{FunctionalDependencyPredicate.py}.

\insertcodefile{FunctionalDependencyPredicate.py}{Instantiation of FunctionalDependencyPredicate}

\begin{description}
\item [alpha] attributes being the alpha of the given functional dependency. Given as either a single attribute name, or a tuple of attribute names.
\item [beta] attributes which are functionally dependent on alpha. Given as either a single attribute name, or a tuple of attribute names.
For example with alpha as \texttt{('a', 'b')} and beta as  \texttt{'c'} we get the functional dependency: $a,\ b \rightarrow c$.
\end{description}

In our instantiation above, we use the predicate to assert that there is a functional dependency between a book's title and its author. If this assertion holds, it means that no book is written by more than one author.

In \Cref{SQLFunctionalDependency.sql} we see the query generated in this case. Here we join the table with itself on alpha, then return all distinct rows in which alpha can not be used to determine beta. If the query returns any rows, the assertion fails and the rows are reported to the tester.    

\insertcodefileSQL{SQLFunctionalDependency.sql}{SQL query generated from
\Cref{FunctionalDependencyPredicate.py}}
