\subsection{ReferentialIntegrityPredicate}
ReferentialIntegrityPredicate asserts that all table entries in all tables in the given datawarehouse connection, have referential integrity. This means that each row of a table with a foreign key, has a corresponding row in the table to which it is referring and vice versa. A user may always want to have full integrity, and this is a good way to check that integrity has not been violated. This predicate takes no parameters, it is enough to simply create one instance and use it for any number of cases and with any number number of datawarehouse connections.

ReferentialIntegrityPredicate makes use of the DWRepresentation object's \emph{refs} field, which is similar to \emph{refs} used in pygramETL's snowflaked dimensions. \emph{refs} is a dictionary with referring tables as keys and lists of tables as values. ReferentialIntegrityPredicate uses this to make a dictionary of dictionaries that holds more information on the references. Now these inner dictionaries will have foreign keys as their key with the referenced table as the value. In this way it is easy to look up what the foreign keys are called and what they refer to. It is then quick to iterate over the tables, their rows, and comparing their keys that they have in common. After iterating between a referring table and a referred table, the checks are made again by iterating over the referred table's rows first. This is done because the new \emph{refs} have no information on the tables a table is referred by, only what it refers to. If the predicate fails to find a corresponding id for a row in another table, it takes note of the two tables and the key for reporting. Currently works with facttables and snowflaked dimensions.