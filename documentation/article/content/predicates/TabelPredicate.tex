\subsection{TabelPredicate}
TabelPredicate is a variation of the RuleColumnPredicate in \cref{RCP}. The difference being that TabelPredicate allows the user check their constraint against an the entire column at once, instead of going trough each row as RuleColumnPredicate does. The input to arguments of the constraint is by default a list, each representing the column(s) in the order they are specified. However if return\_list is set to false, the input for the constraint is either the sum if the column contains integers, or the join of all the strings if the column contains strings. TabelPredicate is instantiated as follows, with description of the parameters below

\insertcodefile{codeRelated/scripts/TabelPredicate.py}{Instantiation of TabelPredicate}

\begin{description}
\item [table\_name] a string containing the name of specific table from as data warehouse. 
\item [constraint\_function] a predicate that represent the constraint which need to be tested. When called it must return either true, if the constraint is fulfilled, or false if it's not.
\item [column\_names] a list of strings or a string containing the name(s) of the column(s), which specifies which column(s) the predicate should test within in the table
\item [return\_list] a bool, if false then all elements from the column(s) is send as arguments to the constraint function, if true they are send as a list instead. Default is True.
\end{description}

TabelPredicate iterates over all the rows in specified table
