\subsection{RuleRowPredicate}\label{RCP}
RuleRowPredicate is used to assert that every row of a table complies with some user-defined constraint. Thus, this predicate allows users a lot of flexibility in how they test business rules for individual rows through python code. The RuleRowPredicate may be instantiated as shown in \Cref{RuleRowPredicate.py}.

\insertcodefile{RuleRowPredicate.py}{Instantiation of RuleRowPredicate}

\begin{description}
\item [constraint\_function] a python function that represents the user constraint. It must return a boolean, indicating whether a given row conforms to the constraint.
\item [constraint\_args] a list of additional arguments given to the constraint\_function.
\end{description}

In the instantiation \Cref{RuleRowPredicate.py}, we use the no\_autobios as our constraint function. The predicate as a whole asserts that no author in the DW has written a book with their own name as the title. Autobiographies are generally expected to be of poor quality, and we do not want them in our DW. 

With column\_names we define, which row attributes that constraint\_function should get as input. As we iterate over each row, we call constraint\_function. The function receives the defined row attributes as parameters along with those from constraint\_args. If the function returns false for a row, the assertion did not hold and all faulty rows are reported to the tester. Should be noted that all parts of this check are done at python level. This means that we need to fetch all data from the DW at once. Thus this will be slower to use than other predicates.
