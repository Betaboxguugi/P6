\subsection{RuleRowPredicate}\label{RCP}
RuleRowPredicate is used to assert that every row of a table complies with some user-defined constraint. Thus, this predicate allows users a lot of flexibility in how they test business rules for individual rows. The RuleRowPredicate is instantiated as shown below:

\insertcodefile{RuleRowPredicate.py}{Instantiation of RuleRowPredicate}

The parameters comprise of:
\begin{description}
\item [constraint\_function] a function that represent the user constraint. It must return a boolean, indicating whether a given row conforms to the constraint.
\item [constraint\_args] a list of additional arguments given to the constraint\_function.
\end{description}

With column\_names we define, which row attributes that constraint\_function should get as input. As we iterate over each row, we call constraint\_function. The function receives the defined row attributes as parameters along with those from constraint\_args. If the function returns false for a row, the assertion did not hold.

In the instantiation above, we use the initials\_id as our constraint function. Using the predicate as a whole, we assert that every author's primary key corresponds to their initials. 
