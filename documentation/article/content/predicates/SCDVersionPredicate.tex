\subsection{SCDVersionPredicate}\label{SCD}
This predicate allows testers to check if a given entry in a table has an asserted largest version. This predicate only works on SCDType2Representation objects, which represent type 2 slowly changing dimensions. SCDVersionPredicate is instantiated with parameters as depicted in \Cref{SCDVersionPredicate.py}.

\insertcodefile{SCDVersionPredicate.py}{Instantiation of RuleColumnPredicate}

\begin{description}
\item [entry] A dictionary that pairs lookup-attribute names with values. A set of lookup-attributes being a key on the table, it is used to fetch all instances of unique row under assertion.
\item [version] The asserted largest version value of the test entry.
\end{description}

In the instantiation, we use the predicate to assert that The Hobbit's highest version number in BookDim is 10.

The predicate works through the query shown in \Cref{SQLSCDVersion.sql}. Here it uses \textit{entry} to fetch all instances of the entry in \textit{table\_name}. Each instance having a different version number. The query then extracts the maximum value of the version attribute. The name of version-attribute is fetched from the metadata in the table's corresponding DimRepresentation object. In this case the name is simply 'version'.

\insertcodefileSQL{SQLSCDVersion.sql}{SQL query generated from \cref{SCDVersionPredicate.py}}

After having fetched the actual maximum value, it is compared to the one asserted by the tester. If the two values are equal, the assertion holds. Otherwise it fails.