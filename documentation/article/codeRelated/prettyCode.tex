\usepackage{listings} % Required for inserting code snippets

%Changing lis name from 'Listing' to 'Code Example'
\renewcommand{\lstlistingname}{Code Example}

%Defining colors used in our style
\definecolor{color3}{RGB}{230, 231, 231} % background
\definecolor{Blue}{RGB}{0, 51, 204} % keywords
\definecolor{Gray}{RGB}{153, 153, 153} % line numbers
\definecolor{PineGlade}{RGB}{204, 204, 153}

%Our style
\lstdefinestyle{Python1}{ 
language=Python,
frame=single,
basicstyle=\scriptsize\ttfamily,
backgroundcolor=\color{color3},
keywordstyle=\color{color1}\bf,
captionpos=b,
breakatwhitespace=false,
breaklines=true,
numbers=left, % Location of line numbers, can take the values of: none, left, right
numbersep=6pt, % Distance of line numbers from the code box
numberstyle=\tiny\color{Gray}, % Style used for line numbers
commentstyle=\usefont{T1}{pcr}{m}{sl}\color{color1}
} 

\newcommand{\insertcodefile}[2]{\lstinputlisting[caption=#2,label=#1,style=Python1,float=h]{#1}} % The first argument is the script location/filename and the second is a caption for the code

\newcommand{\insertcodefileline}[4]{\begin{itemize}\item[]\lstinputlisting[firstnumber=#3,firstline=#3,lastline=#4,caption=#2,label=#1,style=Python1]{#1}\end{itemize}}


%----------------------------------------------------------------------------------------
%ALDRIG KALD DENNE STYLE!!!!!!!!!!!!!!!!!!!!!, kun her som reference til hvordan vi laver vores egen.
%----------------------------------------------------------------------------------------

\lstdefinestyle{Style1}{ % Define a style for your code snippet, multiple definitions can be made if, for example, you wish to insert multiple code snippets using different programming languages into one document
language=Python, % Detects keywords, comments, strings, functions, etc for the language specified
backgroundcolor=\color{highlight}, % Set the background color for the snippet - useful for highlighting
basicstyle=\footnotesize\ttfamily, % The default font size and style of the code
breakatwhitespace=false, % If true, only allows line breaks at white space
breaklines=true, % Automatic line breaking (prevents code from protruding outside the box)
captionpos=b, % Sets the caption position: b for bottom; t for top
commentstyle=\usefont{T1}{pcr}{m}{sl}\color{DarkGreen}, % Style of comments within the code - dark green courier font
deletekeywords={}, % If you want to delete any keywords from the current language separate them by commas
%escapeinside={\%}, % This allows you to escape to LaTeX using the character in the bracket
%firstnumber=1, % Line numbers begin at line 1
frame=single, % Frame around the code box, value can be: none, leftline, topline, bottomline, lines, single, shadowbox
frameround=tttt, % Rounds the corners of the frame for the top left, top right, bottom left and bottom right positions
keywordstyle=\color{Blue}\bf, % Functions are bold and blue
morekeywords={}, % Add any functions no included by default here separated by commas
numbers=left, % Location of line numbers, can take the values of: none, left, right
numbersep=10pt, % Distance of line numbers from the code box
numberstyle=\tiny\color{Gray}, % Style used for line numbers
rulecolor=\color{black}, % Frame border color
showstringspaces=false, % Don't put marks in string spaces
showtabs=false, % Display tabs in the code as lines
stepnumber=1, % The step distance between line numbers, i.e. how often will lines be numbered
stringstyle=\color{Purple}, % Strings are purple
tabsize=2, % Number of spaces per tab in the code
}
