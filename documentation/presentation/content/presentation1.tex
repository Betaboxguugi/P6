\section{Introduction}

\begin{frame}{Problem}{}
  Hvad vil vi?
  \begin{itemize}
  \item Vi vil lave et framework som kan hjælpe ETL programmører med at teste deres systemer
  \end{itemize}
\end{frame}

\begin{frame}{Problem}{}
  Det nuværende marked
  \begin{itemize}
  \item<2-> Table comparisons
    \begin{itemize}
    \item e.g. AnyDBTest
    \item Pro: Folk kan lave assertions omkring stort set alt
    \item Con: Kræver meget kodning, hvor man nemt kan lave fejl
    \end{itemize}
  \item<3-> GUI baseret testing
    \begin{itemize}
    \item e.g. QuerySurge
    \item Pro: Kræver ikke meget kode
    \item Con: GUI baseret og kan hurtigt blive kompleks.
    \end{itemize}
  \end{itemize}
\end{frame}


\begin{frame}{Problem}{}
  Kriterier til vores framework
  \begin{itemize}
  \item<2-> Frameworket skal kunne bruges til automation af tests
    \begin{itemize}
    \item Da agilt er vejen frem og automation af tests er en hjørne sten deri
    \end{itemize}
  \item<3-> Frameworket skal mindske det krævede kode som skal skrives for at udføre ens tests
    \begin{itemize}
    \item Mindre test kode leder som udgangspunkt til mindre bugs i ens tests
      \begin{itemize}
      \item Nuværende test software kræver typisk meget kode i form af at sætte tables op
      \end{itemize}
    \end{itemize}
  \item<4-> Det skal være kode orienteret
    \begin{itemize}
    \item Samme filosofi som pygrametl
    \end{itemize}
  \end{itemize}
\end{frame}

\begin{frame}{SkiRaff}{}
  SkiRaff
  \begin{itemize}
  \item<2-> Et framework til at teste ETL programmer
  \item<3-> Man laver assertions om ens populated DW ved hjælp af Predicates
    \begin{itemize}
      \item Disse Predicates modelere typiske ting som man vil teste for og kan tilpasses til ens DW
    \end{itemize}
  \item<4-> Kan lave funktionelle tests på et system niveau
    \begin{itemize}
    \item Pro: Vi tester systemet som en helhed, og kan fange fejl som er skyldet af at flere komponeneter interagere med hinanden
    \item Con: Gør at det er svært at finde ud af præcis hvor fejl opstår
    \end{itemize}
  \item<5-> Funktionalitet til at man kan udskifte data kilder til test data kilder
    \begin{itemize}
    \item Hvis man bruger pygrametl
    \end{itemize}
  \item<6-> Bygget til at kunne samarbejde med pygrametl
    \begin{itemize}
    \item Kan dog sagtens bruges uden
    \end{itemize}
  \item<7-> Kan bruges sammen med PEP249 compatible DBMS'er
  \end{itemize}
\end{frame}

\begin{frame}{SkiRaff}{}
  Overview af frameworkets komponenter \\ \\7
  [Lav en fin graf her!]
\end{frame}

\begin{frame}{Demo}{}
Demo Af SkiRaff
\end{frame}
