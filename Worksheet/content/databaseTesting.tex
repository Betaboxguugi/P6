\section{Database Testing}

\subsection{The importance of testing databases.}

As with any form of software, it is important to insure that as few bugs as possible are present within it. But with databases, testing is more critical from a business perspective, as a crash or faulty features, could insure heavy losses for both the company and customers.

\subsection{Key components of databases that should be tested:}

\subsubsection{Data Mapping:}

When interacting with a database, it is usually through a UI of some description. So we should insure that the data which comes from the database is displayed correctly, as well as the data which is written into the UI is sent to the correct part of the database. Also all actions done from the UI, has some corresponding action in the database, for example create, delete or update. These actions should therefore also be tested to insure the right actions are executed upon the database.

\subsubsection{ACID:}
ACID is an acronym which stands for atomicity, consistency, isolation and durability. All transactions in a database must adhere to these concepts, to insure accuracy and stability in a database. \cite{ACID} \\
\textbf{Atomicity} is the all or nothing rule. A entire transactions must either fail or pass, so if a part of a transaction fails, the entire transaction fails. \\
\textbf{Consistency} is that any transaction will always result in a valid state. It helps to insure that any changes to values in a transactions are also consistent with changes to other values in the same transaction. \\
\textbf{Isolation }is if 2 or more transaction are executed at once, then the result from this should remain the same as if the transaction were executed one after the other. \\
\textbf{Durability} is that any power loss or crash of the database, should be unable to change the data in the database.

\subsection{Methods to test databases:}

Black box testing and white box testing(also know as Clear-box) are the 2 main methods for testing databases.
When working with databases though, it is important to insure that the database has the right setup before testing it. Therefore it is important to create a Test Data Generator, so one can easily create the proper database for a given test case.

%TODO: Include picture databaseTesting

\subsubsection{White box testing:} 
White box testing is testing of the internal structure of the database, where the items being tested are known to the tester.  Commonly the following is tested on databases when performing white box tests:
\begin{itemize}
	\item Module testing of database functions, triggers, views, SQL queries etc.
	\item Validating of database tables, data models, database schema etc.
	\item Insure referential integrity
	\item The ACID concepts
\end{itemize}

\subsubsection{Black box testing:}
Black box testing is testing of the external structure of the database, such as the interface and integration of the database. Here the tester does not know the items being tested. Commonly the following is tested on databases when performing black box tests:

\begin{itemize}
	\item Mapping of data (including metadata)
	\item Verifying incoming data
	\item Verifying outgoing  data(queries, views, stored functions)
\end{itemize}